\documentclass[11pt]{article}
\usepackage[margin=1in]{geometry}
\usepackage{amssymb,amsmath,amsfonts}
\usepackage[none]{hyphenat}
\usepackage{fancyhdr}
\usepackage{mdwlist}


\pagestyle{fancy}

\fancyhead{}
\fancyfoot{}
\fancyhead[L]{\slshape \MakeUppercase{Assignment 3}}
\fancyhead[R]{\slshape Valentina Sanchez}
\fancyfoot[C]{\thepage}
\renewcommand{\headrulewidth}{0pt}%remove horizontal line

\begin{document}

\begin{titlepage}
	\begin{center}

	\vspace*{1cm}
		\Large{\textbf{Algorithms and Data Structures}}\\
		%\Large{\textbf{February 22nd, 2021}}\\
		\vfill
		\line(1,0){400}\\ [3mm]
		\Huge{\textbf{Assignment 3}}\\ [3mm]
		\Large{\textbf{- \today -}}\\ [1mm]
		\line(1,0){400}\\
		\vfill
		Camila Santana \\
		
	\end{center}
\end{titlepage}

\section{Problem 3.1}
	\subsection{$f(n)=9n \ , \ g(n) = 5n^3$}
		\begin{enumerate}
			\item $f \in O(g)$ \\
				In order for $f \in O(g)$, $\lim_{x \to \infty} \frac{f(n)}{g(n)} < \infty$	must hold. \\
				$\lim_{x \to \infty} \frac{9n}{5n^3} = 0 < \infty$ \\
				$\therefore f \in O(g)$
				
			\item $f \in \Omega(g)$\\
				In order for $f \in \Omega(g)$, $\lim_{x \to \infty} \frac{f(n)}{g(n)} > 0$	must hold. \\
				$\lim_{x \to \infty} \frac{9n}{5n^3} = 0 $ , it is not $>$ 0 \\
				$\therefore f \notin \Omega(g)$
				
			\item $f \in \Theta(g)$ \\
				In order for $f \in \Theta(g)$, $0 < \lim_{x \to \infty} \frac{f(n)}{g(n)} < \infty$	must hold. \\
				$\Theta(g) = O(g) \cap \Omega(g)$ \\
				It was proven that $f \notin \Omega(g)$ \\
				$\therefore f \notin \Theta(g)$
			
			\item $f \in o(g)$ \\
				In order for $f \in o(g)$, $\lim_{x \to \infty} \frac{f(n)}{g(n)} = 0$	must hold. \\
				$\lim_{x \to \infty} \frac{9n}{5n^3} = 0 $\\
				$\therefore f \in o(g)$
			
			\item $f \in \omega(g)$ \\
				In order for $f \in \omega(g)$, $\lim_{x \to \infty} \frac{f(n)}{g(n)} = \infty$	must hold. \\
				$\lim_{x \to \infty} \frac{9n}{5n^3} = 0 \neq \infty$ \\
				$\therefore f \notin \omega(g)$
			
			
			\item $g \in O(f)$ \\
				As claimed previously $ f \notin \Omega(g)$ \\ 
				According to the transpose symmetric property $f(n) = O(g(n)) \Leftrightarrow g(n) = \Omega (f(n)) $ \\
				$\therefore g \notin O(f)$
				
			\item $g \in \Omega(f)$ \\
				As claimed previously $ f \in O(g)$ \\ 
				According to the transpose symmetric property $f(n) = \Omega (g(n)) \Leftrightarrow g(n) = O(f(n)) $ \\
				$\therefore g \in \Omega (f)$
			
			\item $g \in \Theta(f)$ \\
				It was previously proven that $g \notin O(f)$ \\
				$\Theta(f) = O(f) \cap \Omega(g)$\\
				$\therefore g \notin \Theta (g)$
				
			\item $g \in o(f)$ \\
				In order for $f \in o(f)$, $\lim_{x \to \infty} \frac{g(n)}{f(n)} = 0$	must hold. \\
				$\lim_{x \to \infty} \frac{5n^3}{9n} = \infty $\\
				$\therefore g \notin o(f)$
				
			\item $g \in \omega(f)$ \\
				In order for $f \in o(f)$, $\lim_{x \to \infty} \frac{g(n)}{f(n)} = \infty$	must hold. \\
				$\lim_{x \to \infty} \frac{5n^3}{9n} = \infty $\\
				$\therefore g \in \omega(f)$
		\end{enumerate}
		
	%%%%%%%%%%%%%%%%%%%%%%%%%%%%%%
		
	\subsection{$f(n)=9n^{0.8} + 2n^{0.3} + 14log(n) \ , \ g(n) = n^{0.5}$} 
		\begin{enumerate}
			\item $f \in O(g)$ \\
				In order for $f \in O(g)$, $\lim_{x \to \infty} \frac{f(n)}{g(n)} < \infty$	must hold. \\
				$\lim_{x \to \infty} \frac{9n^{0.8} + 2n^{0.3} + 14log(n)}{n^{0.5}} = \infty$ \\
				$\therefore f \notin O(g)$
				
			\item $f \in \Omega(g)$\\
				In order for $f \in \Omega(g)$, $\lim_{x \to \infty} \frac{f(n)}{g(n)} > 0$	must hold. \\
				$\lim_{x \to \infty} \frac{9n^{0.8} + 2n^{0.3} + 14log(n)}{n^{0.5}} = \infty > 0$ \\
				$\therefore f \in \Omega(g)$
				
			\item $f \in \Theta(g)$ \\
				In order for $f \in \Theta(g)$, $0 < \lim_{x \to \infty} \frac{f(n)}{g(n)} < \infty$	must hold. \\
				$\Theta(g) = O(g) \cap \Omega(g)$ \\
				It was proven that $f \notin O(g)$\\
				$\therefore f \notin \Theta(g)$
			
			\item $f \in o(g)$ \\
				In order for $f \in o(g)$, $\lim_{x \to \infty} \frac{f(n)}{g(n)} = 0$	must hold. \\
				$\lim_{x \to \infty} \frac{9n^{0.8} + 2n^{0.3} + 14log(n)}{n^{0.5}} = \infty $\\
				$\therefore f \notin o(g)$
			
			\item $f \in \omega(g)$ \\
				In order for $f \in \omega(g)$, $\lim_{x \to \infty} \frac{f(n)}{g(n)} = \infty$	must hold. \\
				$\lim_{x \to \infty} \frac{f9n^{0.8} + 2n^{0.3} + 14log(n)}{n^{0.5}} = \infty$ \\
				$\therefore f \in \omega(g)$
			
			
			\item $g \in O(f)$ \\
				As claimed previously $ f \in \Omega(g)$ \\ 
				According to the transpose symmetric property $f(n) = O(g(n)) \Leftrightarrow g(n) = \Omega (f(n)) $ \\
				$\therefore g \in O(f)$
				
			\item $g \in \Omega(f)$ \\
				As claimed previously $ f \notin O(g)$ \\ 
				According to the transpose symmetric property $f(n) = \Omega (g(n)) \Leftrightarrow g(n) = O(f(n)) $ \\
				$\therefore g \notin \Omega (f)$
			
			\item $g \in \Theta(f)$ \\
				It was previously proven that $g \notin \Omega(f)$ \\
				$\Theta(f) = O(f) \cap \Omega(g)$\\
				$\therefore g \notin \Theta (g)$
				
			\item $g \in o(f)$ \\
				In order for $f \in o(f)$, $\lim_{x \to \infty} \frac{g(n)}{f(n)} = 0$	must hold. \\
				$\lim_{x \to \infty} \frac{n^{0.5}}{9n^{0.8} + 2n^{0.3} + 14log(n)} = 0 $\\
				$\therefore g \in o(f)$
				
			\item $g \in \omega(f)$ \\
				In order for $f \in o(f)$, $\lim_{x \to \infty} \frac{g(n)}{f(n)} = \infty$	must hold. \\
				$\lim_{x \to \infty} \frac{n^{0.5}}{9n^{0.8} + 2n^{0.3} + 14log(n)} = 0 $\\
				$\therefore g \notin \omega(f)$
		\end{enumerate}
		
		%%%%%%%%%%%%%%%%%%%%%%%%%%%%%%%
		
	\subsection{$f(n)=\frac{n^2}{log(n)} \ , \ g(n) = nlog(n)$}
		\begin{enumerate}
			\item $f \in O(g)$ \\
				In order for $f \in O(g)$, $\lim_{x \to \infty} \frac{f(n)}{g(n)} < \infty$	must hold. \\
				We simplify our initial given functions to $\lim_{x \to \infty} \frac{n}{(log(n))^{2}} = \infty$ \\
				$\therefore f \notin O(g)$
				
			\item $f \in \Omega(g)$\\
				In order for $f \in \Omega(g)$, $\lim_{x \to \infty} \frac{f(n)}{g(n)} > 0$	must hold. \\
				$\lim_{x \to \infty} \frac{n}{(log(n))^{2}} = \infty > 0$ \\
				$\therefore f \in \Omega(g)$
				
			\item $f \in \Theta(g)$ \\
				In order for $f \in \Theta(g)$, $0 < \lim_{x \to \infty} \frac{f(n)}{g(n)} < \infty$	must hold. \\
				$\Theta(g) = O(g) \cap \Omega(g)$ \\
				It was proven that $f \notin O(g)$\\
				$\therefore f \notin \Theta(g)$
			
			\item $f \in o(g)$ \\
				In order for $f \in o(g)$, $\lim_{x \to \infty} \frac{f(n)}{g(n)} = 0$	must hold. \\
				$\lim_{x \to \infty} \frac{n}{(log(n))^{2}} = \infty $\\
				$\therefore f \notin o(g)$
			
			\item $f \in \omega(g)$ \\
				In order for $f \in \omega(g)$, $\lim_{x \to \infty} \frac{f(n)}{g(n)} = \infty$	must hold. \\
				$\lim_{x \to \infty} \frac{n}{(log(n))^{2}} = \infty$ \\
				$\therefore f \in \omega(g)$
			
			
			\item $g \in O(f)$ \\
				As claimed previously $ f \in \Omega(g)$ \\ 
				According to the transpose symmetric property $f(n) = O(g(n)) \Leftrightarrow g(n) = \Omega (f(n)) $ \\
				$\therefore g \in O(f)$
				
			\item $g \in \Omega(f)$ \\
				As claimed previously $ f \notin O(g)$ \\ 
				According to the transpose symmetric property $f(n) = \Omega (g(n)) \Leftrightarrow g(n) = O(f(n)) $ \\
				$\therefore g \notin \Omega (f)$
			
			\item $g \in \Theta(f)$ \\
				It was previously proven that $g \notin \Omega(f)$ \\
				$\Theta(f) = O(f) \cap \Omega(g)$\\
				$\therefore g \notin \Theta (g)$
				
			\item $g \in o(f)$ \\
				In order for $f \in o(f)$, $\lim_{x \to \infty} \frac{g(n)}{f(n)} = 0$	must hold. \\
				$\lim_{x \to \infty} \frac{(log(n))^{2}}{n} = 0 $\\
				$\therefore g \in o(f)$
				
			\item $g \in \omega(f)$ \\
				In order for $f \in o(f)$, $\lim_{x \to \infty} \frac{g(n)}{f(n)} = \infty$	must hold. \\
				$\lim_{x \to \infty} \frac{(log(n))^{2}}{n} = 0 $\\
				$\therefore g \notin \omega(f)$
		\end{enumerate}
		
		%%%%%%%%%%%%%%%%%%%%%%%%%%%%%%%%%
	\subsection{$f(n)=(log(3n))^3 \ , \ g(n) = 9log(n)$}
		\begin{enumerate}
			\item $f \in O(g)$ \\
				In order for $f \in O(g)$, $\lim_{x \to \infty} \frac{f(n)}{g(n)} < \infty$	must hold. \\
				We simplify our initial given functions to $\lim_{x \to \infty} \frac{(log(3n))^3}{9log(n)} = \infty$ \\
				$\therefore f \notin O(g)$
				
			\item $f \in \Omega(g)$\\
				In order for $f \in \Omega(g)$, $\lim_{x \to \infty} \frac{f(n)}{g(n)} > 0$	must hold. \\
				$\lim_{x \to \infty} \frac{(log(3n))^3}{9log(n)} = \infty > 0$ \\
				$\therefore f \in \Omega(g)$
				
			\item $f \in \Theta(g)$ \\
				In order for $f \in \Theta(g)$, $0 < \lim_{x \to \infty} \frac{f(n)}{g(n)} < \infty$	must hold. \\
				$\Theta(g) = O(g) \cap \Omega(g)$ \\
				It was proven that $f \notin O(g)$\\
				$\therefore f \notin \Theta(g)$
			
			\item $f \in o(g)$ \\
				In order for $f \in o(g)$, $\lim_{x \to \infty} \frac{f(n)}{g(n)} = 0$	must hold. \\
				$\lim_{x \to \infty} \frac{(log(3n))^3}{9log(n)} = \infty $\\
				$\therefore f \notin o(g)$
			
			\item $f \in \omega(g)$ \\
				In order for $f \in \omega(g)$, $\lim_{x \to \infty} \frac{f(n)}{g(n)} = \infty$	must hold. \\
				$\lim_{x \to \infty} \frac{(log(3n))^3}{9log(n)} = \infty$ \\
				$\therefore f \in \omega(g)$
			
			
			\item $g \in O(f)$ \\
				As claimed previously $ f \in \Omega(g)$ \\ 
				According to the transpose symmetric property $f(n) = O(g(n)) \Leftrightarrow g(n) = \Omega (f(n)) $ \\
				$\therefore g \in O(f)$
				
			\item $g \in \Omega(f)$ \\
				As claimed previously $ f \notin O(g)$ \\ 
				According to the transpose symmetric property $f(n) = \Omega (g(n)) \Leftrightarrow g(n) = O(f(n)) $ \\
				$\therefore g \notin \Omega (f)$
			
			\item $g \in \Theta(f)$ \\
				It was previously proven that $g \notin \Omega(f)$ \\
				$\Theta(f) = O(f) \cap \Omega(g)$\\
				$\therefore g \notin \Theta (g)$
				
			\item $g \in o(f)$ \\
				In order for $f \in o(f)$, $\lim_{x \to \infty} \frac{g(n)}{f(n)} = 0$	must hold. \\
				$\lim_{x \to \infty} \frac{9log(n)}{(log(3n))^3} = 0 $\\
				$\therefore g \in o(f)$
				
			\item $g \in \omega(f)$ \\
				In order for $f \in o(f)$, $\lim_{x \to \infty} \frac{g(n)}{f(n)} = \infty$	must hold. \\
				$\lim_{x \to \infty} \frac{9log(n)}{(log(3n))^3} = 0 $\\
				$\therefore g \notin \omega(f)$
		\end{enumerate}


	
		
	
	

	


\end{document}















